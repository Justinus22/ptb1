\documentclass[../main.tex]{subfiles}
\begin{document}

%In den \glspl{CCC} geht es darum - wie durch die Bezeichnung bereits vorweggenommen - die Compliance von Code festzustellen. Allerdings wurde noch nicht geklärt, was damit gemeint ist.

Code gilt als compliant, wenn er alle ihn betreffenden Richtlinien erfüllt.
Das können Richtlinien von Staatlichen Institutionen wie der \acrshort{EU} oder Deutschland sein, Richtlinien, die für das ganze Unternehmen SAP gelten, aber auch Richtlinien, die nur für bestimmte Abteilungen gelten.

Deshalb ist für die \glspl{CCC} wichtig zu wissen, wie die Richtlinien von SAP genutzt werden können.
Diese umfassen unter anderem Regeln zum Qualitätsmanagement, zur Export-Kontrollklassifizierung, Opensource Lizenzvalidierung und zu diversen Überprüfungen der Sicherheitsanforderungen.

\end{document}
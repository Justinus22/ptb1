\documentclass[../main.tex]{subfiles}
\begin{document}

In den \glspl{CCC} geht es darum - wie durch die Bezeichnung bereits vorweggenommen - die Compliance von Code festzustellen. Allerdings wurde noch nicht geklärt, was damit gemeint ist.

Wenn Code compliant ist, heißt das, dass der Code alle ihn betreffenden Richtlinien erfüllt.
Das können Richtlinien von Staatlichen Institutionen wie der \acrshort{EU} oder Deutschland sein, Richtlinien, die für das ganze Unternehmen SAP gelten, aber auch Richtlinien die nur für bestimmte Abteilungen gelten.

Deshalb ist für diese Arbeit interessant, wie die Richtlinien von SAP genutzt werden können.
Diese umfassen unter anderem Regeln zum Qualitätsmanagement, zur Export Kontrollklassifizierung, Opensource Lizenzvalidierung, diverse Überprüfungen der Sicherheitsanforderungen und vieles mehr.


\end{document}


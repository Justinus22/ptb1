\documentclass[../main.tex]{subfiles}
\begin{document}

Im Unternehmen SAP SE besteht die Notwendigkeit, dass alle Software-Entwickler die globale Entwicklungsrichtlinie von SAP und insbesondere die in dieser Richtlinie festgelegten Regeln befolgen. 
Die Einhaltung dieser Regeln und damit die Einhaltung der SAP Global Development-Richtlinie für alle Entwicklungseinheiten ist ungemein wichtig und obligatorisch, um jegliche rechtlichen und finanziellen Risiken, sowie Ansehensverluste für das Unternehmen zu vermeiden.

Die SAP-Abteilung \gls{CSI} unterstützt Service- und Produktteams beim Programmieren, Versenden und Ausführen ihrer Services und Produkte in der Cloud.
Sie bietet ein Portfolio von Diensten, die eine Plattform für die Entwicklung, Veröffentlichung und den Betrieb von Cloud-nativen, konformen und produktionsbereiten Diensten und Anwendungen bilden.

Die Dienste zur Überprüfung der SAP-Entwicklungsrichtlinien werden vom Team \gls{HHCQE} entwickelt und betreut.
Mit sogenannten \glspl{CCC} werden die Richtlinien in automatisierten Entwicklungs- und Bereitstellungsprozessen überprüft und die SAP-konforme Auslieferung neuer oder aktualisierter Software, z. B. zur Fehlerbehebung, gewährleistet. 

In diesem Praxistransferbericht wird erarbeitet, wie Checks gestaltet werden, damit sie alle genannten Anforderungen erfüllen und gleichzeitig so minimalistisch bleiben, dass kein Entwickler mehr als nötig durch negative Check Ergebnisse aufgehalten wird. 

\end{document}
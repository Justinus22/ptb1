\documentclass[../main.tex]{subfiles}
\begin{document}

Abschließend lässt sich festhalten, dass die praktisch umgesetzte Implementierung alle theoretisch erarbeiteten Anforderungen erfüllt.
D. h., dass innerhalb einer \gls{glos:pipeline} durch einen Upload zu einem Hyperspace-Server und einem anschließenden Download der Auswertungsergebnisse entschieden werden kann, ob der Code compliant ist oder nicht.
Der Entscheidungsfindungsprozess wird durch von dafür Verantwortlichen in dem unmissverständlichen \gls{glos:PolicyAsCode} Format deutlich gemacht.
Für tiefere Einblicke kann in den genutzten Scan-Tools eine Weboberfläche genutzt werden, die alle Risikostellen dokumentiert und einordnet.

So kann der rasante \gls{CI/CD}-Entwicklungsprozess für die SAP-Cloud abgesichert werden, damit SAP Kunden durchgehend hochqualitative Produkte erhalten.

Trotzdem bleiben noch Fragen offen.
Was machen die Teams, die eigene strengere oder mildere Anforderungen haben?
Gibt es gegebenenfalls Richtlinien, die gar nicht zentral in einem Policy As Code Format dargelegt werden können?
Wie sieht die Bewertung mit Richtlinien aus, wenn von einem Scan keine eindeutige \gls{glos:JSON}-Ergebnisdatei erzeugt wird?

Wie zu erkennen, lässt dieses Feld nach ersten Anstrengungen weiterhin viel Raum zur Entwicklung und Einbindung von verbesserten \acrlongpl{CCC} in der SAP-Softwareentwicklung.

\end{document}



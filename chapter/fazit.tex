\documentclass[../main.tex]{subfiles}
\begin{document}

Abschließend lässt sich festhalten, dass die umgesetzte Implementierung alle erarbeiteten Anforderungen erfüllt.
D.h., dass innerhalb einer \gls{glos:pipeline} durch einen Upload zu einem Hyperspace Server und einem anschließenden Download der Auswertungsergebnisse entschieden werden kann, ob Code compliant oder nicht ist.
Der Entscheidungsfindungsprozess wird durch von dafür Verantwortlichen in dem unmissverständlichen \gls{glos:PolicyAsCode} Format deutlich gemacht.
Für tiefere Einblicke kann in den genutzten Scan Tools eine Weboberfläche genutzt werden, die alle Risikostellen dokumentiert und einordnet.

Trotzdem bleiben noch Fragen offen.
Was machen die Teams die eigene strengere oder laschere Anforderungen haben?
Gibt es auch Richtlinien, die man gar nicht zentral in einem Policy As Code Format darlegen kann?
Wie sieht die Bewertung mit Richtlinien aus, wenn von einem Scan keine eindeutige \gls{glos:JSON}-Ergebnisdatei erzeugt wird?


\end{document}



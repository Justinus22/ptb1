
\documentclass[
	12pt, %Schriftgröße
	a4paper,
	liststotoc, %Inhaltsverzeichniseinträge für Listen (z.B. Abbildungen)
	bibtotoc, %Inhaltsverzeichniseinträge f+r Quellen
	pointlessnumbers, %Entfernt Punkt hinter Gliederungsnummern
	ngerman, %Sprachpaket
	headsepline, %Headertrennlinie
	%footsepline, %Footertrennlinie
	oneside %einseitiges Druckformat %%% Unterdrücken der leeren Seite nach Titelblatt
	]{scrbook} %Dokumentenklasse (Koma-Script)
\usepackage[T1]{fontenc}
\usepackage{float}
\usepackage[utf8]{inputenc}
\usepackage[ngerman]{babel}
\usepackage{url}
\usepackage{graphicx} %Bilder einfügen
%\usepackage{pdfpages} %PDF einfügen
\usepackage[a4paper, margin=1in]{geometry}
\usepackage[right]{eurosym} %Euro-Zeichen
\usepackage{amssymb}
\usepackage{cite} %Quellenangaben
\usepackage{setspace} % Zeilenabstand
\usepackage[ 
   colorlinks,        % Links ohne Umrandungen in zu wählender Farbe 
   linkcolor=black,   % Farbe interner Verweise 
   filecolor=black,   % Farbe externer Verweise 
   citecolor=black,   % Farbe von Zitaten 
   urlcolor=blue	  % Farbe von Links
   ]{hyperref} %Verlinkungen
\usepackage[figure]{hypcap}
\usepackage[ngerman]{translator}
\usepackage{blindtext} % Lorem-Ipsum-Plugin
\usepackage[acronym, nonumberlist]{glossaries} %% use after hyperref %Glossar-Paket laden
%\usepackage[
	%nonumberlist, %keine Seitenzahlen anzeigen
	%acronym,      %ein Abkürzungsverzeichnis erstellen
	%toc,          %Einträge im Inhaltsverzeichnis
	%section      %im Inhaltsverzeichnis auf section-Ebene erscheinen
	%]
%{glossaries}

\usepackage{listings,xcolor} %Codeanzeige
\usepackage[normalem]{ulem}
\useunder{\uline}{\ul}{}

\usepackage{chngcntr}
\counterwithout{figure}{chapter}
\counterwithout{table}{chapter}

\definecolor{dkgreen}{rgb}{0,.6,0}
\definecolor{dkblue}{rgb}{0,0,.6}
\definecolor{dkyellow}{cmyk}{0,0,.8,.3}

\lstset{
    numbers=left, 
    numberstyle=\tiny, 
    numbersep=5pt,
    breaklines=true,
    frame=single,
    escapeinside={(*@}{@*)}, %nicht anzuzeigende Ausdrücke, z.B. für Labels
    language=sh,
    basicstyle=\ttfamily\fontsize{10}{12}\selectfont,
    keywordstyle    = \color{dkblue},
    stringstyle     = \color{red},
    identifierstyle = \color{dkgreen},
    commentstyle    = \color{gray},
    emph            =[1]{php},
    emphstyle       =[1]\color{black},
    emph            =[2]{if,and,or,else},
    emphstyle       =[2]\color{dkyellow}
    } 

%%%%%%%%%%%%%%%%%%%%%%%%%%%%%%%%%%%%%%%%%%%%%%%%%%%%%
%%%%%%%%%%% Sonderformatierung
%%%%%%%%%%%%%%%%%%%%%%%%%%%%%%%%%%%%%%%%%%%%%%%%%%%%%

% Seitenabstände definieren
\geometry{verbose,tmargin=3cm,bmargin=2cm,lmargin=3cm,rmargin=3cm} 

% Hurenkinder und Schusterjungen verhindern (Ja, das heißt wirklich so!!!)
\clubpenalty = 10000 \widowpenalty = 10000 \displaywidowpenalty = 10000 

\newcommand{\footfigref}[1]{\footnote{Abb. \ref{#1} auf Seite \pageref{#1}}}

%% Bei Referenzen im Text wird jetzt bei allen Ebenen "Kapitel" vorgestellt, z.b. Kapitel 2, Kapitel 2.2, Kapitel 6.3.2
\addto\extrasngerman{%
    \def\sectionautorefname{Kapitel}%
    \def\subsectionautorefname{Kapitel}%
    \def\subsubsectionautorefname{Kapitel}%
    }

% Vertikaler Abstand zwischen Ende Textblock - Ende Fußzeile --> Abstand der Seitenzahl von Rand erhöhen 
\setlength{\footskip}{10mm}

% Abstand vor/nach Überschriften verändern

\RedeclareSectionCommand[%
    beforeskip=0.5\baselineskip,
    afterskip=0.5\baselineskip
]{chapter}

\RedeclareSectionCommand[%
    beforeskip=0.5\baselineskip,
    afterskip=0.5\baselineskip
]{section}

\RedeclareSectionCommand[%
    beforeskip=0.1\baselineskip,
    afterskip=0.1\baselineskip
]{subsection}

\RedeclareSectionCommand[%
    beforeskip=0.01\baselineskip,
    %%afterskip=0.2\baselineskip
]{paragraph}

\setlength{\abovecaptionskip}{4pt}  % 1pc=12pt 
\setlength{\belowcaptionskip}{0pt}
%\setlength{\textfloatsep}{4pt}
\setlength{\intextsep}{1pc}

%% Verkleinerung der Textgröße unter Abbildungen
\addtokomafont{caption}{\small}

% falsche Default-Silbentrennung überschreiben
\include{hyphenation}

% Den Punkt am Ende der Glossareinträge deaktivieren
\renewcommand*{\glspostdescription}{}

%Glossar-Befehle anschalten
%\makeglossaries

% sorgt dafür, dass bei Leerzeile die Einrückung verhindert und stattdessen eine Leerzeile eingefügt wird % erspart bigskips und erhöht die Lesbarkeit im LaTeX-Text 
\KOMAoptions{parskip=full*}

% ändert Titelschriftart in Serifen-Normalschriftart
\addtokomafont{disposition}{\rmfamily} 

\makenoidxglossaries

\loadglsentries{glossar.tex}


%%%%%%%%%%%%%%%%%%%%%%%%%%%%%%%%%%%%%%%%%%%%%%%%%%%%%
\newcommand{\studentName}{Justin Becker}
\newcommand{\type}{Praxistransferbericht}
\newcommand{\topic}{Einbindung von Code Compliance Checks in der SAP Software Entwicklung}
\newcommand{\studiengangh}{Informatik}
\newcommand{\fachbereich}{FB2: Duales Studium - Technik}
\newcommand{\studiengang}{Informatik}
\newcommand{\company}{SAP SE}
\newcommand{\betreuerHS}{Lara Maria Stricker}
\newcommand{\betreuerUnt}{Jenny Peter}
\newcommand{\jahrgang}{2023}
%%%%%%%%%%%%%%%%%%%%%%%%%%%%%%%%%%%%%%%%%%%%%%%%%%%%%>>>>>>>

\begin{document}

%%%%%%%%%%%%%%%%%%%%%%%%%%%%%%%%%%%%%%%%%%%%%%%%%%%%%>>>>>>>
%%%%%%%%%%% Titelblatt

%% Anordnung und Aussehen von Titel und Untertitel

\subject{\type}

\title{
\normalfont\endgraf\rule{\textwidth}{.4pt}
\begingroup
	\centering
	\linespread{1.5}
	\huge\topic
\endgroup
\endgraf\rule{\textwidth}{.4pt}
}
 
%%Eigentlich nicht besonders schön, aber Koma erlaubt die Anordnung eines weiteren Felder (hier: Fachbereich) nicht
\date{\normalsize vorgelegt am 8. August 9834\\ \textbullet \\ Fachbereich Duales Studium Wirtschaft / Technik \\
Hochschule für Wirtschaft und Recht Berlin}
%% \date muss leer angegeben werden, um die Default-Datumsanzeige zu unterdrücken

\publishers{
	\begin{tabular}{l l}
	\textbf{\normalsize{}} & \normalsize{}  \tabularnewline
	\textbf{\normalsize{}} & \normalsize{}  \tabularnewline
	\textbf{\normalsize{Name:}} & \normalsize{\studentName}  \tabularnewline
	\textbf{\normalsize{Ausbildungsbetrieb:}} & \normalsize{\company}  \tabularnewline
    \textbf{\normalsize{Fachbereich:}} & \normalsize{\fachbereich} \tabularnewline
    \textbf{\normalsize{Studienjahrgang:}} & \normalsize{\jahrgang} \tabularnewline
	\textbf{\normalsize{Studiengang:}} & \normalsize{\studiengang}  \tabularnewline
	\textbf{\normalsize{Betreuerin Unternehmen:}} & \normalsize{\betreuerUnt} \tabularnewline
	\textbf{\normalsize{Betreuerin Hochschule:}} & \normalsize{\betreuerHS}
	\end{tabular}
	}

\titlehead{\begin{center}
    \includegraphics[scale=0.7]{bilder/header_logo.PNG}
    \end{center}
    }

\maketitle

\onehalfspacing % anderthalbfacher Zeilenabstand

%%%%%%%%%%%%%%%%%%%%%%%%%%%%%%%%%%%%%%%%%%%%%%%%%%%%%%%%%%%%%%%%%%%%%%%%%%%%%%%%%%%%%%%%%%%%%%%%%%%%%%%%%%%%%%%%%%%%%%%%%%%
%%%%%%%%%%% Dokumenteninhalt START
%%%%%%%%%%%%%%%%%%%%%%%%%%%%%%%%%%%%%%%%%%%%%%%%%%%%%%%%%%%%%%%%%%%%%%%%%%%%%%%%%%%%%%%%%%%%%%%%%%%%%%%%%%%%%%%%%%%%%%%%%%%

%%%%%%%%%%%%%%%%%%%%%%%%%%%%%%%%%%%%%%%%%%%%%%%%%%%%%
%%%%%%%%%%% Abstract
\chapter*{Zusammenfassung}
\addcontentsline{toc}{chapter}{Abstract}

Im Unternehmen SAP SE besteht die Notwendigkeit, dass alle Software-Entwickler die globale Entwicklungsrichtlinie von SAP und insbesondere die in dieser Richtlinie festgelegten Regeln, die sogenannten Unternehmensanforderungen befolgen.

Die Anforderungen umfassen unter anderem Regeln zum Qualitätsmanagement, zur Export Kontrollklassifizierung, Opensource Lizenzvalidierung und diverse Überprüfungen der Sicherheitsanforderungen.

Die Einhaltung dieser Regeln und damit die Einhaltung der SAP Global Development-Richtlinie für alle Entwicklungseinheiten ist ungemein wichtig und obligatorisch, um jegliche rechtlichen und finanziellen Risiken, sowie Ansehensverluste für das Unternehmen zu vermeiden.

Die SAP-Abteilung „Common Service Infrastructure“ unterstützt Service- und Produktteams beim Programmieren, Versenden und Ausführen ihrer Services und Produkte in der Cloud.
 Es bietet ein Portfolio von Diensten, die eine Plattform für die Entwicklung, Veröffentlichung und den Betrieb von Cloud-nativen, konformen und produktionsbereiten Diensten und Anwendungen bilden.

Die Dienste zur Überprüfung der SAP-Entwicklungsrichtlinien werden vom Team „HANA \& HANA Cloud Quality Engineering“ entwickelt und betreut.
Mit sogenannten Checks und Gates werden die Richtlinien in automatisierten Entwicklungs- und Bereitstellungsprozessen überprüft und die SAP konforme Auslieferung neuer oder aktualisierter Software, z.B. zur Fehlerbehebung, gewährleistet.
 Auf Grund hoher Dynamik in der Cloud Service Entwicklung und dadurch resultierenden geänderten Testanforderungen, sowie neuen oder verbesserten Tools zur Softwareüberprüfung, welche von den Checks angesteuert werden, müssen sowohl neue Checks und Gates etabliert als auch vorhandene Checks angepasst werden.

\pagenumbering{Roman} % römische Seitenzahlen

%%%%%%%%%%%%%%%%%%%%%%%%%%%%%%%%%%%%%%%%%%%%%%%%%%%%%
%%%%%%%%%%% Inhaltsverzeichnis, Tabellen, Abbildungen, etc.
\newpage

\tableofcontents{}
\addcontentsline{toc}{chapter}{Inhaltsverzeichnis}

\listoffigures
\listoftables


\addcontentsline{toc}{chapter}{Akronyme}
\printnoidxglossaries

\clearpage

%% arabische Seitenzahlen
\pagenumbering{arabic}

%%%%%%%%%%%%%%%%%%%%%%%%%%%%%%%%%%%%%%%%%%%%%%%%%%%%%
%%%%%%%%%%% Einführung

\chapter{Allgemeines}\label{sec:allgemein}

Diese Einführung soll einen kurzen Überblick über die allgemeinen Möglichkeiten von \LaTeX{} geben. 
Deshalb hier ein Test.

Es kann auf Bilder wie das HWR-Logo verwiesen werden (s. \autoref{fig:hwrlogo}) oder auf Tabellen (s. \autoref{table:tab_spalten}).

\begin{figure}[h]
  \centering
  \includegraphics[scale=0.8]{bilder/header_logo.png}
  \label{fig:hwrlogo}       %fig:ID
  \caption[HWR-Logo: Überschrift Abbildungsverzeichnis]{HWR-Logo Bildunterschrift}    %Bildunterschrift
\end{figure}

\begin{table}[h]
\centering
\small	
\begin{tabular}{|l l l l|}
\hline
Spalte 1 & Spalte 2 & Spalte 3 & Spalte 4\\
\hline
\end{tabular}
\caption{Eine Tabelle mit Spalten}
\label{table:tab_spalten}
\end{table}

Auch Quellenverweise sind möglich. Quellen werden in der Datei literatur.bib angelegt und tauchen automatisch im Literaturverzeichnis auf, wenn der Text einen entsprechenden Verweis enthält \cite[S. 42-1337]{XP}. Auch Glossareinträge wie \gls{glos:AntwD} oder Abkürzungen wie \gls{DMZ}, \gls{AD} und \gls{CD} folgen dieser Regel.

Verweise auf Kapitel sind ebenfalls möglich. Kapitel werden zu diesem Zweck mit einem Label versehen (s. \autoref{sec:allgemein}).
Außerdem gibt es natürlich so schöne Dinge wie Aufzählungen\footnote{und Fußnoten}

\begin{itemize}
	\item Wenn bloß eine Aufzählung
	\item benötigt wird
\end{itemize}

und Nummerierungen

\begin{enumerate}
	\item Wenn eine Nummerierung 
	\item gewünscht ist
\end{enumerate}

\section{Besonderheit}

Unter dem vorliegenden Kapitel ist eine Besonderheit dieser Vorlage aufgeführt. Aufgrund von Platz- und Übersichtsgründen soll wie in der ursprünglichen Vorlage im Text nur ein abgekürzter Literatureintrag angezeigt werden, wie \cite[S. 42-1337]{XP}, aber im Literaturverzeichnis soll sich der Gesamteintrag weitestgehend an den APA-Richtlinien orientieren. 

Die Formatierung des Literaturverzeichnisses weicht daher vom Standard ab. Die im Paket enthaltene Datei ,,hwrbib.bst'' bietet diese Möglichkeit an.

Hier noch der Trigger für einige Literaturverzeichniseinträge: 

\begin{itemize}
	\item \cite{HUMMWIE_2005}
	\item \cite{DWD_Beau}
	\item \cite{CRISP}
\end{itemize}

%%%%%%%%%%%%%%%%%%%%%%%%%%
% Quellen
%%%%%%%%%%%%%%%%%%%%%%%%%

\bibliography{literatur}

\bibliographystyle{hwrbib}
%% \bibliographystyle{alpha} %% tu es nicht, niemals, das ist eklig, nicht einkommentieren

\chapter*{Ehrenwörtliche Erklärung}
\addcontentsline{toc}{chapter}{Ehrenwörtliche Erklärung}

% Keine Kopf- und Fußzeilen ausgeben
\thispagestyle{empty}
% Aber trotzdem ins Inhaltsverzeichnis aufnehmen
%\addcontentsline{toc}{section}{Eidesstattliche Erklärung}

% Hier der offizielle Text der eidesstattlichen Erklärung
Ich erkläre ehrenwörtlich:
\begin{enumerate}
	\item dass ich meine Bachelor-Thesis selbstständig verfasst habe,
	\item dass ich die Übernahme wörtlicher Zitate aus der Literatur sowie die Verwendung der Gedanken anderer Autoren an den entsprechenden Stellen innerhalb der Arbeit gekennzeichnet habe,
	\item dass ich meine Bachelor-Thesis bei keiner anderen Prüfung vorgelegt habe.
\end{enumerate}
Ich bin mir bewusst, dass eine falsche Erklärung rechtliche Folgen haben wird.
% Etwas Abstand für die Unterschrift
\vspace{2cm}

% Hier kommt die Unterschrift drüber
\begin{tabular}{lp{4em}l} 
 \hspace{5cm}   && \hspace{4cm} \\\cline{1-1}\cline{3-3} 
 Ort, Datum     && \studentName
\end{tabular}

\end{document}


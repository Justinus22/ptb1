\documentclass[../main.tex]{subfiles}
\begin{document}

Damit ergibt sich als Ziel der \glspl{CCC} in den \gls{CI/CD} \glspl{glos:pipeline}, dass sie das Ergebnis liefern sollen, ob der Code, der mit der \gls{glos:pipeline} ausgeliefert werden soll, allen erforderlichen Richtlinien folgt.
Außerdem müssen sie eine Begründung für diese Entscheidung an die Entwickler liefern, damit dieser seinen Code entsprechend verbessern kann.

Dafür sind \glspl{CCC} in zwei Teile aufgeteilt:
\begin{itemize}
    \item ein Scan, der den Code analysiert und feststellt, welche Richtlinien relevante Risikostellen existieren 
    \item die Bewertung der Scan-Ergebnisse nach den Richtlinien
  \end{itemize}

Da die Richtlinien Anforderungen unterschiedlichster Themenbereiche von der Export-Kontrollklassifizierung bis zur Sicherheit enthalten, wird die Gesamtfeststellung - compliant oder nicht - in mehrere Teilchecks unterteilt, sodass ein Check immer nur einen Aspekt der Richtlinien übernimmt. 

\end{document}


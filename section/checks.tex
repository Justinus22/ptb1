\documentclass[../main.tex]{subfiles}
\begin{document}

Damit ergibt sich als Ziel von den \glspl{CCC} in den \gls{CI/CD} \glspl{glos:pipeline}, dass sie das Ergebnis liefern sollen, ob der Code, der mit der \gls{glos:pipeline} ausgeliefert werden soll, allen gegebenen Richtlinien folgt und damit compliant ist.
Außerdem müssen sie eine Begründung für diese Entscheidung an die Entwickler liefern, damit dieser seinen Code entsprechend verbessern kann.
\newpage
Dafür sind \glspl{CCC} in zwei Teile aufgeteilt:
\begin{itemize}
    \item ein Scan, der den Code analysiert und feststellt welche -----Richtlinien relevante Risikostellen------- u.\"a. existieren 
    \item die Bewertung von den Ergebnissen diesen Scans nach den Richtlinien
  \end{itemize}

Da die Richtlinien Anforderungen unterschiedlichster Themenbereich wie zum Beispiel der Export Kontrollklassifizierung bis zu Sicherheit enthalten, wird die Gesamtfeststellung - compliant oder nicht - in mehrere Teilchecks unterteilt, sodass solch ein Check einen Aspekt der Richtlinien übernimmt. 

\end{document}


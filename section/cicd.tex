\documentclass[../main.tex]{subfiles}
\begin{document}

Da die Cloud-Produkte komplett von SAP verwaltet werden, ergibt sich der wichtige Vorteil für das Unternehmen und deren Kunden, dass die Entwicklungsteams bei SAP den Code von Produkten jederzeit anpassen und an den Kunden ausliefern kann.
Somit können schnell Fehler korrigiert, sowie Feedback des Kunden umgesetzt und agil gearbeitet werden.
So entsteht mehr Produktwert für die Kunden und ihre Zufriedenheit steigt.


Dieser Prozess wird \gls{CI/CD} genannt.
Jeder Entwickler integriert kontinuierlich neue Änderungen in den Code, welche dann fortlaufend an den Kunden geliefert werden.
Umgesetzt wird dieses Prinzip durch die Automatisierung mit sogenannten \gls{CI/CD} \glspl{glos:pipeline}.
Diese \glspl{glos:pipeline} übernehmen alle für die Auslieferung nötigen Schritte, wie zum Beispiel die Build-, Test- und Deploy-Vorgänge und ermöglichen damit erst \gls{CI/CD}.
\cite{CI/CD}

Durch den zügigen Entwicklungsprozess ist die Fehleranfälligkeit, d. h. dass ein Code nicht der Compliance entspricht, relativ hoch.
Deshalb ist es ein unerlässlicher Schritt, die Code Compliance vor dem Ausliefern als Teil der \gls{CI/CD} \glspl{glos:pipeline} festzustellen.
Deshalb werden \glspl{CCC} als solch ein Schritt eingeführt.

\end{document}
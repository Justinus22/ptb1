\documentclass[../main.tex]{subfiles}
\begin{document}

Da die Cloud Produkte komplett von SAP verwaltet werden, ergibt sich der wichtige Vorteil für die SAP und deren Kunden, dass die Entwicklungsteams bei SAP den Code von Produkten jederzeit anpassen und an den Kunden ausliefern kann.
Somit können schnell Fehler verbessert, Feedback des Kunden umgesetzt und agil gearbeitet werden.
So entsteht mehr Produktwert für die Kunden und ihre Zufriedenheit steigt.


Diesen Prozess wird \gls{CI/CD} genannt.
Jeder Entwickler integriert kontinuierlich neue Änderungen in den Code, welche dann kontinuierlich an den Kunden geliefert werden.
Umgesetzt wird dieses Prinzip durch die Automatisierung mit sogenannten \gls{CI/CD} \glspl{glos:pipeline}.
Diese \glspl{glos:pipeline} übernehmen alle für die Auslieferung nötigen Schritte wie zum Beispiel die Build-, Test- und Deploy-Vorgänge und ermöglichen damit erst \gls{CI/CD}.
\cite{CI/CD}

Durch den schnellen Entwicklungsprozess entsteht schnell Code der nicht mehr compliant ist.
Deshalb ist es ein unerlässlicher Schritt die Code Compliance vor dem Ausliefern als Teil der \gls{CI/CD} \glspl{glos:pipeline} festzustellen.
Deshalb werden \glspl{CCC} als solch ein Schritt eingeführt.

\end{document}


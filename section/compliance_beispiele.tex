\documentclass[../main.tex]{subfiles}
\begin{document}

Ein konkretes Beispiel Code der ein Sicherheitsrisiko ist und damit eventuell für nicht compliance verantwortlich ist, ist die in Abbildung \ref{fig:codecompliance} dargestellte Funktion.

\begin{figure}[ht]
    \centering
    \includegraphics[scale=1]{bilder/CodeScreenShot.png}
    \caption{Risikocode, da eine geöffnete Datei nicht richtig geschlossen wird (selbst geschrieben)}
    
    \label{fig:codecompliance}
\end{figure}

Solche Fehler könnten zu unerwartetem Verhalten während der Laufzeit führen.
Deshalb müssen sie bereits davor erkannt werden.
Da eine Überprüfung per Hand fehleranfällig und langsam ist, braucht es hierfür eine Automatisierung.
Diese Automatisierung wird Check genannt.
\end{document}



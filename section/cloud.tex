\documentclass[../main.tex]{subfiles}
\begin{document}
Die Strategie der SAP sieht es vor, dass möglichst viele dieser Produkte in einer \gls{glos:PbC} als \gls{SaaS} verfügbar sind \cite{SAPCloudStrategy}.
Die Entwicklung von Cloud Produkten hat unter anderem den Vorteile, dass Kunden nicht in eigen Hardware und Infrastruktur investieren müssen und damit flexibel bleiben.
Da Cloud Produkte meist in einem Abonnement Modell verkauft werden, ist ein Vorteil für die SAP konstante und kalkulierbare Einnahmen.
Deshalb bauen aktuelle SAP Produkte, wie in Abbildung \ref{fig:produktportfolio} zu sehen, auf diesem Prinzip auf.
\cite{CloudProContra}

Damit die Entwicklung der Produkte entsprechend der erklärten Prinzipien abläuft, gibt es Abteilungen, die sich darauf fokussieren den Cloud Entwicklungsprozess zu vereinfachen.
So bietet \gls{CSI} eine einheitliche Infrastruktur für diese Entwicklungsprozess.
So kann als Teil dieser Infrastruktur überprüft werden, ob alle Anforderungen an den Code des Produktes erfüllt sind -- ob der Code compliant ist.

Umgesetzt werden diese Anforderungen als Teil des \gls{CI/CD} Prozesses.
\end{document}
\documentclass[../main.tex]{subfiles}
\begin{document}
Die Strategie der SAP sieht es vor, dass möglichst viele dieser Produkte in einer \gls{glos:PbC} als \gls{SaaS} verfügbar sind \cite{SAPCloudStrategy}.
Die Entwicklung von Cloud Produkten hat unter anderem den Vorteile, dass Kunden nicht in eigen Hardware und Infrastruktur investieren müssen und damit flexibel bleiben.
Da Cloud Produkte meist in einem Abonnement Modell verkauft werden, ist ein Vorteil für die SAP konstante und kalkulierbare Einnahmen.
Deshalb bauen aktuelle SAP Produkte auf diesem Prinzip auf wie in Abbildung \ref{fig:produktportfolio} zu sehen.
\cite{CloudProContra}

Damit die Entwicklung der Produkte reibungslos abläuft, gibt es Abteilungen, die sich darauf fokussieren den Cloud Entwicklungsprozess zu vereinfachen.
So bietet \gls{CSI} eine einheitliche Infrastruktur für diese Entwicklungsprozess. Diese Infrastruktur bietet diese und diese Vorteile…

Die Anforderungen umfassen unter anderem Regeln zum Qualitätsmanagement, zur Export Kontrollklassifizierung, Opensource Lizenzvalidierung und diverse Überprüfungen der Sicherheitsanforderungen.
\end{document}